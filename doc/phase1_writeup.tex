\documentclass{article}

\begin{document}
\title{Phase 1}
\author{Barry Warnock \\
  CPSC 371 \\}
\maketitle{}
\section{The Cube}
A Rubiks Cube is a square with 6 faces each with n x n sections. A section can have one of six colours and each face begins in a different
uniform colour. A face is broken up into n slices on both the X and Y axis. The each one of these slices can be rotated clockwise or counter-clockwise
affecting the sections on both that face and other faces depending on the axis of rotation.
\subsection{Encoding}
I numbered the sections of each face with x,y coordinates starting on the 'upper left' which is defined in such a way so that for each face rotating
to a face with the same parity will not require any extra math to be done to the indices (with the exception to rotations on the Z axis)

The actual structure of the cube in code is a 1d array where each section is indexed by the formula $(n*n*face)+(y*n)+x$
\subsection{Rotations}
Rotations are done by iterating through all of the faces affected by the rotation and for each face setting (to\_face, to\_x(from\_x, from\_y),
to\_y(from\_x, from\_y)) to the value of (from\_face, from\_x, from\_y). where from\_face = the face the sections are coming from, to\_face = the face
the sections are going to, from\_x = the x value on from\_face, from\_y = the y value on from\_face, to\_x and to\_y are lambda functions dependent on
from\_x and from\_y that are set depending on the rotation axis and the equivalent parities of from\_face and to\_face.

\subsubsection{Face Rotations}
When a face needs to be rotated (face, to\_x(from\_y), to\_y(from\_x)) is set to (face, from\_x, from\_y) where to\_x and to\_y are set based on the
direction of rotation.

\section{Classes / Data Structures / Algorithms}
\subsection{Searchable}
\begin{itemize}
\item generate a list of Searchable objects that can be reached from the current state
\item return its state as a string
\item return a heuristic value representing how many steps from solved it thinks it is
\item compare itself to other Searchables of the same sub-type
\item return a string representing itself in a human readable format
\end{itemize}
\subsubsection{Cube}
described in section 1.

\subsection{Search\_Node}
A Search\_Node is a struct with the following data fields:
\begin{itemize}
\item a pointer to the Search\_Node representing the Searchable it came from
\item a pointer to the Searchable it represents
\item a string representing the move that was taken to get to it from the parent
\item the depth of this Search\_Node from the start state
\item a guess about how many steps it is from solved
\end{itemize}

\subsection{Search}
Search takes two Searchable objects a start and a goal and returns the Search\_Node representing solved state. This can be used to get the full
path from start to finish.
\subsubsection{Breadth First Search}
BFS works by checking every possible child state of each state then doing the same to each of those children and so on until a solved state is
reached.
\subsubsection{A* Search}
A* works like BFS but instead of ordering the children to check next by the distance from start they are ordered by a guess about how close
they are to solved.
\end{document}
